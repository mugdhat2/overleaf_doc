\documentclass{article}
\usepackage[utf8]{inputenc}
\usepackage{amsfonts,amsmath, amsthm}
\usepackage{geometry}
\usepackage{url}
\usepackage{appendix}
\usepackage{hyperref}
\usepackage{caption}
\usepackage{subcaption}
\usepackage{graphicx}
\usepackage[dvipsnames]{xcolor}
\geometry{margin = 1.25in}
\title{\textit{dismolib}: Free database of user-interactive mathematical models for infectious diseases}
\author{}
\date{}

\usepackage{natbib}
\usepackage{graphicx}

\begin{document}

\maketitle

\section{Introduction}
Compartmental SIR or SEIR models have been used extensively to study the mechanisms and dynamics of a disease for evaluating the effect of public health interventions and policies. Hundreds of compartmental model based infectious disease papers are often published every year for given diseases. With so many efforts using the same basic model type, for the same disease, and running the same types of analyses, a lot of duplicated effort is inevitable. This also adds to the time and effort consumers of the literature need to spend in finding new information about a popularly modeled disease. In this work, we use well-established tools and standards to provide base models for common classes of diseases, using relevant modern disease examples. The tools also provide, built-in, powerful analysis functionality. Our idea and hope is to make it easy for a researcher to start with an existing model, rather than creating one from scratch. This derived model ought to then be easier for others researcher to test, modify, or extend further, because it will be captured in a standard transfer format.

\section{Curation process}
\begin{itemize}
    \item Select disease modeling papers with a simple minimal model for a given example disease of a type.
    \item Perform and document local stability analysis.
    \item Encode the model in COPASI software so that it reproduces figures in the source paper.
    \item Upload the code to the web application created using RShiny, COPASI, ShinyCopasi and CoRC.
\end{itemize}


\section{Features}
\begin{enumerate}
    \item On the web application, users can vary the parameter values corresponding to the interventions to observe changes to the time course, stability analysis, sensitivity analysis, optimization and parameter estimation.
    \item Access to detailed documentation of each disease with mathematical analysis and potential extensions to the model (\url{http://dismolib.uvadcos.io/}).
    \item Option to download the code file (COPASI or SBML format).
    \item Option to easily download the results of the simulations.
\end{enumerate}


\section{Usage and Utility }
The following summarizes how different classes of users might find this library and associated tools useful, as well as the general benefit.
\begin{enumerate}
    \item \textbf{Novice modelers and students}: The database provides a great set of examples for simple models for beginners. Additionally, it requires no coding experience to generate results based on the models. It also provides a great initial source of literature for different types of models and interventions.
    \item \textbf{Epidemiologists}: With merely a few clicks, one can demonstrate the effect of various interventions for different diseases without requiring any coding, mathematical or computational experience.
    \item \textbf{Mathematical modelers}: Quick and ready to use stability analysis and sensitivity analysis programs provide an easy way to validate the results. Importantly, the COPASI based models perform these tasks fast and easily even for tediously big systems (with a large number of compartments).
    \item \textbf{Computational modelers}: Quick calibration, optimization and estimation of parameters is possible with a variety of methods to choose from. Extending a model to include any time-varying parameters only needs a file upload. 
    \item \textbf{Published modelers}: The curated models in the database will be used for future research increasing their visibility and utility.
\end{enumerate}

Requests can be made for a particular model to be curated and/or included as well as for extensions to the existing models. We aim at making this database a collaborative project by infectious disease modelers. Rather than simply being a(nother) repository for models, the curation process also provides a reproduce-ability check for published results.

\section{Future Direction}
\begin{enumerate}
    \item Include within host models
    \item Automate adding spatial component to models
    \item Tutorials demonstrating how different factors can easily be added and removed in a given model using COPASI
    \item Stochastic models
    \item Tutorial videos for using COPASI for disease models and for using the web application
    \item Options to export to MATLAB
    
\end{enumerate}
\end{document}