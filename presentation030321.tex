\documentclass{beamer}
%
% Choose how your presentation looks.
%
% For more themes, color themes and font themes, see:
% http://deic.uab.es/~iblanes/beamer_gallery/index_by_theme.html
%
\mode<presentation>
{
  \usetheme{default}      % or try Darmstadt, Madrid, Warsaw, ...
  \usecolortheme{default} % or try albatross, beaver, crane, ...
  \usefonttheme{default}  % or try serif, structurebold, ...
  \setbeamertemplate{navigation symbols}{}
  \setbeamertemplate{caption}[numbered]
} 

\usepackage[english]{babel}
\usepackage[utf8]{inputenc}
\usepackage[T1]{fontenc}

\title[]{\includegraphics[width = 0.4\textwidth]{dismolib.png}\\ A Library of Infectious Diseases Models}
\author{Mugdha Thakur\\
Brian Klahn\\
Stefan Hoops\\
Baltazar Espinoza}
\institute{\includegraphics[width = 0.4\textwidth]{inst_bicomplex_4c_c.jpg}}
\date{}

\begin{document}

\begin{frame}
  \titlepage
\end{frame}

% Uncomment these lines for an automatically generated outline.
%\begin{frame}{Outline}
%  \tableofcontents
%\end{frame}

\section{Introduction}

\begin{frame}{Background}

\begin{itemize}
  \item Lots of versions of models for any infectious disease exist (since a century now)
  \item Researchers have to start from scratch to build a model and perform analytical exercises
  \item Using parameter values from different sources for the same model impacts the time course simulation of the disease

 
\end{itemize}

\vskip 1cm

\begin{block}{Examples}
<Include examples of number of models for a disease with same structure>
\end{block}

\end{frame}

\section{Highlights}

%\subsection{Tables and Figures}

\begin{frame}{Highlights}

\begin{itemize}
\item Computational and mathematical epidemiology tools for beginners
\item Replication of the (simplest) models with vital characteristics for 10 diseases
\item Easy addition of intervention(s) for the diseases in the model
\item Remark on uniqueness/specialty of each model
\item Remark on reproducibility (pointing at commonly missed information necessary for reproducing results)
\end{itemize}

% Commands to include a figure:
%\begin{figure}
%\includegraphics[width=\textwidth]{your-figure's-file-name}
%\caption{\label{fig:your-figure}Caption goes here.}
%\end{figure}


\end{frame}



\begin{frame}{Next steps}

\begin{itemize}
    \item Web interface for the ease of use: Interactive modeling for parameters of choice, option to download the model to edit locally, option to submit a model.
\item  Build a library of region-wise parameter sets for each disease (for various administrative levels)
\begin{itemize}
    \item E.g., National level parameter set for the US for all diseases
\end{itemize}
\item  Build an extension for each disease model with additional important factors
\begin{itemize}
    \item E.g., seasonality in COVID-19 (done), sex-segregated model for STDs, age-structured model for  Measles, metapopulation model for emerging diseases, etc.
\end{itemize}
\item  Time-dependent sensitivity analysis of each parameter (specially when it is an arbitrary assumption)

\end{itemize}

\end{frame}

\begin{frame}{Examples}
    
\end{frame}

\begin{frame}{Resources}
    \begin{itemize}
        \item Webapp: \url{ http://dismolib.uvadcos.io/}
        \item \textbf{dismolib} repository: \url{https://github.com/mugdhat2/CopasiDiseaseLibrary}
        \item Documentation (in progress) of diseases, models and simulations: \url{https://www.overleaf.com/read/tsyhztjmhdfv}
        \item Get COPASI: \url{copasi.org}
    \end{itemize}
\end{frame}

\end{document}
