\documentclass[10pt,letterpaper]{article}
\usepackage[top=0.85in,left=2.75in,footskip=0.75in]{geometry}
\usepackage[utf8x]{inputenc}
\usepackage{setspace}
\usepackage{changepage}
\usepackage{amsfonts,amsmath, amsthm}
%\usepackage{geometry}
\usepackage{url}
\usepackage{appendix}
\usepackage{hyperref,nameref}
% textcomp package and marvosym package for additional characters
\usepackage{textcomp,marvosym}

% cite package, to clean up citations in the main text. Do not remove.
\usepackage{cite}

\usepackage{array}
% create "+" rule type for thick vertical lines
\newcolumntype{+}{!{\vrule width 2pt}}

% create \thickcline for thick horizontal lines of variable length
\newlength\savedwidth
\newcommand\thickcline[1]{%
  \noalign{\global\savedwidth\arrayrulewidth\global\arrayrulewidth 2pt}%
  \cline{#1}%
  \noalign{\vskip\arrayrulewidth}%
  \noalign{\global\arrayrulewidth\savedwidth}%
}

% \thickhline command for thick horizontal lines that span the table
\newcommand\thickhline{\noalign{\global\savedwidth\arrayrulewidth\global\arrayrulewidth 2pt}%
\hline
\noalign{\global\arrayrulewidth\savedwidth}}
% ligatures disabled
\usepackage{microtype}
\DisableLigatures[f]{encoding = *, family = * }
\usepackage{subcaption}
\doublespacing
% Text layout
\raggedright
\setlength{\parindent}{0.5cm}
\textwidth 5.25in 
\textheight 8.75in

% Bold the 'Figure #' in the caption and separate it from the title/caption with a period
% Captions will be left justified
\usepackage[aboveskip=1pt,labelfont=bf,labelsep=period,justification=raggedright,singlelinecheck=off]{caption}
\renewcommand{\figurename}{Fig}
\usepackage{graphicx}
\usepackage[right]{lineno}
\usepackage[table]{xcolor}
\makeatletter
\renewcommand{\@biblabel}[1]{\quad#1.}
\makeatother



% Header and Footer with logo
\usepackage{lastpage,fancyhdr,graphicx}
\usepackage{epstopdf}
%\pagestyle{myheadings}
\pagestyle{fancy}
\fancyhf{}
%\setlength{\headheight}{27.023pt}
%\lhead{\includegraphics[width=2.0in]{PLOS-submission.eps}}
\rfoot{\thepage/\pageref{LastPage}}
\renewcommand{\headrulewidth}{0pt}
\renewcommand{\footrule}{\hrule height 2pt \vspace{2mm}}
\fancyheadoffset[L]{2.25in}
\fancyfootoffset[L]{2.25in}
\lfoot{\today}


\title{Parameter estimation of the 2015-2016 Zika outbreaks for the differences in the geopolitical regions of Brazil using COPASI}
\author{}
\date{}


\begin{document}
\vspace*{0.2in}

% Title must be 250 characters or less.
\begin{flushleft}
{\Large
\textbf\newline{Parameter estimation of the 2016 Zika outbreaks for the differences in the geopolitical regions of Brazil using COPASI} % Please use "sentence case" for title and headings (capitalize only the first word in a title (or heading), the first word in a subtitle (or subheading), and any proper nouns).
}
\newline
% Insert author names, affiliations and corresponding author email (do not include titles, positions, or degrees).
\\
Mugdha Thakur\textsuperscript{1*},
Baltazar Espinoza\textsuperscript{1},
Brian Klahn\textsuperscript{1},
Stefan Hoops\textsuperscript{1}
% Name5 Surname\textsuperscript{2\ddag},
% Name6 Surname\textsuperscript{2\ddag},
% Name7 Surname\textsuperscript{1,2,3*},
% with the Lorem Ipsum Consortium\textsuperscript{\textpilcrow}
\\
\bigskip
\textbf{1} Biocomplexity Institute and Initiative, University of Virginia, Charlottesville, Virginia, USA
\\
% \textbf{2} Affiliation Dept/Program/Center, Institution Name, City, State, Country
% \\
% \textbf{3} Affiliation Dept/Program/Center, Institution Name, City, State, Country
% \\
\bigskip

% Insert additional author notes using the symbols described below. Insert symbol callouts after author names as necessary.
% 
% Remove or comment out the author notes below if they aren't used.
%
% Primary Equal Contribution Note
%\Yinyang These authors contributed equally to this work.

% Additional Equal Contribution Note
% Also use this double-dagger symbol for special authorship notes, such as senior authorship.
%\ddag These authors also contributed equally to this work.

% Current address notes
%\textcurrency Current Address: Dept/Program/Center, Institution Name, City, State, Country % change symbol to "\textcurrency a" if more than one current address note
% \textcurrency b Insert second current address 
% \textcurrency c Insert third current address

% Deceased author note
%\dag Deceased

% Group/Consortium Author Note
%\textpilcrow Membership list can be found in the Acknowledgments section.

% Use the asterisk to denote corresponding authorship and provide email address in note below.
* mat3kk@virginia.edu

\end{flushleft}
% Please keep the abstract below 300 words
\section*{Abstract}



% Please keep the Author Summary between 150 and 200 words
% Use first person. PLOS ONE authors please skip this step. 
% Author Summary not valid for PLOS ONE submissions.   
\section*{Author summary}

\linenumbers
%\maketitle

%\begin{abstract}
    
%\end{abstract}


\section*{Introduction}
Zika virus (ZIKV) disease, although symptomatically mild, may cause severe morbidity due to microcephaly and other congenital and pregnancy problems, along with the suspected cases of Gullian-Barre Syndrome, brain and spinal cord swelling and blood disorders \cite{paixao2016history,da2017neurologic}. ZIKV, a flavivirus, can spread by the bite of \textit{Aedes} mosquitoes, through sexual contact and blood transfusions and can be passed to the fetus from an infected pregnant woman. ZIKV currently has no treatment or vaccine. Critical associated morbidity and multiple transmission pathways lead to ZIKV's controls to heavily rely on the population and the vector management.

The first confirmed case of ZIKV in the Americas was reported in Brazil in early 2015. Since then, it quickly spread to almost all the nations in the Americas leading to an international public health emergency \cite{heukelbach2016zika}. During this emergency, along with vector control and birth control, mathematical modeling to guide interventions was a major control and prevention strategy used \cite{lowe2018zika}.

Dynamical modeling-based parameter estimation for infectious diseases provide a reliable tool for measuring the force of infection while accounting for the underlying mechanisms of the disease transmission \cite{overton2020using}. It is useful in identifying spatial patterns for the potency of the disease and for calibrating the optimal effect of potential interventions. However, computation of parameter estimates may get tedious due to the requirement of the understanding of mathematics, optimization, programming and the cost of software \cite{hernandez2018parameter}. COPASI \cite{hoops2006copasi} is a freely available, open source and user-friendly software developed with the aim of making the advanced modeling techniques available and accessible to researchers who may not necessarily be experts in programming or numerical methods \cite{bergmann2016piecewise}. Therefore, the Parameter Estimation tool in COPASI will be used in this paper. 

In this retrospective study, we highlight the importance of the spatial resolution for estimating the parameters used in the mathematical models and the complexity that it entails \cite{thomas2016quantifying}. In summary, using a deterministic mathematical model and the region-wise weekly Brazil ZIKV outbreak data, (a) we demonstrate how COPASI, a biochemical pathway simulator, can be easily adapted to obtain accurate parameter estimates very quickly, (b) we estimate the disease transmission coefficients for each geopolitical region of Brazil, (b) compare the estimates with those for the whole nation of Brazil, and, (c) identify the critical characteristics of the regions correlating with the estimates through a sensitivity analysis. 


% \paragraph{Research Questions}
% \begin{enumerate}
%     \item What are the transmission coefficients for the 2016 Zika outbreak in different regions of Brazil?
%     \item What is the effect of spatial resolution on parameter estimates? 
%     \begin{enumerate}
%         \item Calibrate under or over estimation of transmission probabilities as compared to the estimates from overall cases of Brazil
%         \item What factor characterizes the above? (Hypothesis: population density)
%     \end{enumerate}
%     \item (Future) What are the minimum intervention efforts ($u_1$, $u_2$, $u_3$) to prevent outbreaks in each region? (That is to make $R_0$ or $R_{effective}$ less than 1)
    
% \end{enumerate}
\section*{Materials and methods}
In this section we present the mathematical model for Zika used, the available Zika prevalence data for each geopolitical region of Brazil, the computation of the parameter estimates and the local sensitivity analysis for analyzing the effect of various socio-economic and ecological characteristics on the obtained estimates.
\subsection*{Disease model}
We use a compartmental vector-host model similar to previously published models \cite{bonyah2017theoretical,moreno2017role,suparit2018mathematical,dantas2018calibration}. The model takes into account the human to human infection as well as the vector (mosquito) to human transmission. The total human population, $N_{H}(t),$ is divided into the following classes: susceptible humans $S_{H}(t),$ exposed humans $E_{H}(t),$ infected humans $I_{H}(t),$ and recovered humans $R_{H}(t),$ so that $N_{H}(t)=S_{H}+E_{H}+I_{H}+R_{H}$. The mosquito population, $N_{V}(t),$ is partitioned into susceptible vector $S_{V}(t),$ exposed vector $E_{V}(t)$ and infected mosquito $I_{V}(t)$. Thus, $N_{V}=S_{V}+E_{V}+I_{V}$.

The model considers the processes of birth and death of both humans and mosquitoes, infection transmission to humans due to infected humans and infected mosquitoes, infection transmission to the mosquitoes due to infected humans, extrinsic and intrinsic incubation in vectors and humans respectively, and recovery in humans. The rates for all these processes are described along with point estimates in the Table \ref{tab:zika_params}. The model is described by the following system of equations

% \begin{equation} \label{eq:zika_model}
% \begin{array}{l}
% \frac{d}{d t} S_{h}=\Lambda_{h}-(1-u_1)\beta_{h} S_{h}\left(I_{V}+\rho I_{h}\right)-\mu_{h} S_{h} \\
% \frac{d}{d t} E_{h}=(1-u_1)\beta_{h} S_{h}\left(I_{V}+\rho I_{h}\right)-\left(\mu_{h}+\chi_{h}\right) E_{h} \\
% \frac{d}{d t} I_{h}=\chi_{h} E_{h}-\left(\mu_{h}+\gamma(1+\tilde{u_2})\right) I_{h} \\
% \frac{d}{d t} R_{h}=\gamma (1+\tilde{u_2}) I_{h}-\mu_{h} R_{h} \\[2ex]
% \frac{d}{d t} S_{V}=\Lambda_{V}-(1-u_1)\beta_{V} S_{V} I_{h}-\mu_{V}(1+\tilde{u_3}) S_{V} \\
% \frac{d}{d t} E_{V}=(1-u_1)\beta_{V} S_{V} I_{h}-\left(\mu_{V}(1+\tilde{u_3})+\delta_{V}\right) E_{V} \\
% \frac{d}{d t} I_{V}=\delta_{V} E_{V}-\mu_{V}(1+\tilde{u_3}) I_{V}
% \end{array}
% \end{equation}

\begin{equation} \label{eq:zika_model}
\begin{array}{l}
\frac{d}{d t} S_{h}=\Lambda_{h}-\beta_{h} S_{h}\left(I_{V}+\rho I_{h}\right)-\mu_{h} S_{h} \\
\frac{d}{d t} E_{h}=\beta_{h} S_{h}\left(I_{V}+\rho I_{h}\right)-\left(\mu_{h}+\chi_{h}\right) E_{h} \\
\frac{d}{d t} I_{h}=\chi_{h} E_{h}-\left(\mu_{h}+\gamma\right) I_{h} \\
\frac{d}{d t} R_{h}=\gamma I_{h}-\mu_{h} R_{h} \\[2ex]
\frac{d}{d t} S_{V}=\Lambda_{V}-\beta_{V} S_{V} I_{h}-\mu_{V} S_{V} \\
\frac{d}{d t} E_{V}=\beta_{V} S_{V} I_{h}-\left(\mu_{V}+\delta_{V}\right) E_{V} \\
\frac{d}{d t} I_{V}=\delta_{V} E_{V}-\mu_{V} I_{V}
\end{array}
\end{equation}

%Where, $\tilde{u_2} = \frac{u_2}{1-u_2}$ and $\tilde{u_3} = \frac{u_3}{1-u_3}$ with $0 \leq u_1, u_2, u_3 \leq 1$ such that $u_1 = u_2 = u_3 = 0$ imply no interventions.
We express the human-to-human transmission coefficient ($\rho \beta_h$) with respect to the vector-to-human transmission ($\beta_h$).

\begin{table}[!ht]
    \begin{adjustwidth}{-2.25in}{0in} % Comment out/remove adjustwidth environment if table fits in text column.
    \centering
    \caption{\textbf{Parameter values and descriptions. $\rho$ is the ratio of the vector-to-human and the human-to-human transmission rates in \cite{gao2016prevention}. Recovery period ($1/\gamma$) is the sum of the acute and the convalescent infection durations in \cite{gao2016prevention}.} }
    \begin{tabular}{cp{3in}cc}\hline
      Parameter   & Description & Value & Reference \\ \hline
       $1/\mu_h$  & Average human lifespan in Brazil (2016) &75.23 years& \url{worldbank.org}\\ 
       $1/\mu_V$ & Average mosquito lifespan & 14 days&\cite{gao2016prevention}\\
       $\rho$ & Human-to-human disease transmission coefficient relative to the human-to-vector one & 0.25 &\cite{gao2016prevention}\\
       $1/\chi_h$ & Average intrinsic disease incubation period in humans &5 days&\cite{gao2016prevention}\\
       $1/\gamma$ & Average natural recovery rate of humans &25 days&\cite{gao2016prevention}\\
    %   $u_1$ & Scaling parameter (Reduction in disease transmission coefficient due to bednets)&[0,1]&-\\
    %   $u_2$ & Scaling parameter (Increased rate of recovery from treatment relative to natural recovery & [0,1]&-\\
    %   $u_3$ & Scaling parameter (Increased mosquito mortality rate due to insecticide relative to natural mortality & [0,1] &-\\
       $1/\delta_V$ & Average disease incubation period in mosquitoes & 10 days & \cite{gao2016prevention}\\
       $\Lambda_h$ & Total human birth rate & $\mu_hN_h(0)$&Assumed\\
       $\Lambda_V$ & Total mosquito birth rate & Estimated &-\\
       $\beta_h$ & Vector-to-human disease transmission coefficient & Estimated &-\\
       $\beta_V$ & Human-to-vector disease transmission coefficient & Estimated & -\\
       $m$ & Initial ratio of the number of the mosquitoes to the humans & 5 & \cite{gao2016prevention}\\
       \hline
       
    \end{tabular}
    
    \label{tab:zika_params}
    \end{adjustwidth}
\end{table}

\subsection*{Local stability analysis}
We use the Next Generation Matrix method to calculate the basic reproduction number $R_0$ for Brazil and for each of its regions. We assume that the total human and mosquito populations are at the steady state and thus it can be easily shown that $N_h = \Lambda_h/\mu_h$ and $N_V = \Lambda_V/\mu_V$ at equilibrium.

The disease free equilibrium (DFE) is given by $\{N_h,0,0,0,N_V,0,0\}$ for $\{S_h, E_h, I_h, R_h, S_V, E_V, I_V\}$ set of variables, and the components of the next generation matrix are
$$
F=\left(\begin{array}{cccc}
0 & \frac{\rho \beta_{h} \Lambda_{h}}{\mu_{h}} & 0 & \frac{\beta_{h} \Lambda_{h}}{\mu_{h}} \\
0 & 0 & 0 & 0 \\
0 & \frac{\beta_{v} \Lambda_{v}}{\mu_{V}} & 0 & 0 \\
0 & 0 & 0 & 0
\end{array}\right), V=\left(\begin{array}{cccc}
k_{1} & 0 & 0 & 0 \\
-\chi_{h} & k_{2} & 0 & 0 \\
0 & 0 & k_{3} & 0 \\
0 & 0 & -\delta_{V} & \mu_{V}
\end{array}\right)
$$
where $k_{1}=\mu_{h}+\chi_{h}, k_{2}=\left(\mu_{h}+\gamma\right)$ and $k_{3}=\left(\mu_{V}+\delta_{V}\right)$. The basic reproductive number of model is the spectral radius of the matrix $FV^{-1}$
$$\mathcal{R}_{0}=\frac{\rho \beta_{h} \Lambda_{h} \chi_{h}}{2 \mu_{h} k_{1} k_{2}}+\sqrt{\frac{\rho^{2} \beta_{h}^{2} \Lambda_{h}^{2} \chi_{h}^{2}}{4 \mu_{h}^{2} k_{1}^{2} k_{2}^{2}}+\frac{\beta_{h} \Lambda_{h} \chi_{h} \beta_{V} \delta_{V} \Lambda_{V}}{\mu_{h} \mu_{V}^{2} k_{1} k_{2} k_{3}}}.$$



\subsection*{Parameter estimation}
The  Ministry of Health/SVS (Notifiable Diseases Information System) of Brazil has publicly available record of weekly Zika cases by year for each geopolitical region of Brazil \cite{data}.
We estimate the values of the vector-to-human and the human-to-vector transmission coefficients ($\beta_h$ and $\beta_V$ respectively) for each region of Brazil as well as for all of the Brazil using the recursive Differential Evolution method using the COPASI 4.30 software (\url{copasi.org}). The estimates are highly sensitive to the quantities initial condition ($E_H(0)$) and the vector recruitment rate ($\Lambda_V$), the values for which are not known, therefore, we estimate those too simultaneously. The best-fit vector density for each region is calculated as a product of the steady-state vector population size ($N_V$)  divided by the area of the region in square kilometers.  

We assume that all the parameters that are not being estimated are fixed and are same for all the regions. The assumptions used for initial values are as reported in Table \ref{tab:zika_init}.
% \begin{enumerate}
% %    \item No interventions (that is, $u_1 = u_2 = u_3 = 0$)
%     \item Initial total human population: Region-wise 2016 population estimates as shown in Table \ref{tab:zika_pop}.
%     \item Initial total vector population: Ten times human population
%     \item Initially infected humans: Initial value from Brazil outbreak data (Table \ref{tab:zika_data})
%     \item Initially infected vectors: Twice the initially infected humans
% \end{enumerate}

\begin{table}[]
    \centering
     \caption{\textbf{Initial conditions}}
    \begin{tabular}{lc}\hline
        State Variable & Initial value  \\\hline
        Total human population $N_h(0)$    & As in Table \ref{tab:zika_pop} \\
        Total vector population $N_V(0)$ & $mN_h(0)$\\
        Infectious humans $I_h(0)$ & Week 0 in Table \ref{tab:zika_data}\\
        Infectious vectors $I_V(0)$ & $mI_h(0)$\\
        Exposed humans $E_h(0)$ & Estimated \\
        Exposed vectors $E_V(0)$ & 0\\
        Recovered humans $R_h(0)$ & 0\\
        Recovered vectors $R_V(0)$ & 0\\
        Susceptible humans $S_h(0)$ & $N_h(0) - E_h(0) - I_h(0) - R_h(0)$\\
        Susceptible vectors $S_V(0)$ & $N_V(0) - E_V(0) - I_V(0) - R_V(0)$\\\hline
    \end{tabular}
   
    \label{tab:zika_init}
\end{table}


\section*{Results}
% \begin{table}[!ht]
%     \centering
%     \caption{\textbf{Best fit parameter estimates of disease transmission coefficients $\beta_h$ (vector-to-human; rate per mosquito per day) and $\beta_V$ (human-to-vector; ) by region} }
%     \begin{tabular}{|l|c|c|} \hline
%      \textbf{Region }  &  \textbf{$\beta_h$ estimate (S.D.)} & \textbf{$\beta_V$ estimate (S.D.)} \\ \hline
%      North    & 2.4219e-12 (7.32373e-14) & 7.99256e-05 (2.68471e-05)\\\hline
%      Northeast &1.22726e-12 (1.26273e-14) & 2.84033e-05 (2.78074e-06)\\\hline
%      Southeast&7.07004e-10 (6.64855e-11) & 1.6035e-09 (1.80321e-10)\\\hline
%      South& 3.80726e-13 (5.40485e-15)& 1.8008e-04 (5.30581e-05)\\\hline
%      Central West&2.31863e-09 (2.49295e-09)& 1.09722e-08 (1.38842e-08)\\\hline \hline
%      Brazil & 2.48068e-10 (4.68419e-14)& 1.15922e-09 (6.97177e-11)\\\hline
%     \end{tabular}
%      \begin{flushleft}
%       Contact authors for detailed reports
%      \end{flushleft}
    
%     \label{tab:zika_param_est}
% \end{table}

\begin{table}[!ht]
\begin{adjustwidth}{-2.25in}{0in}
    \centering
    \caption{\textbf{Best fit parameter estimates of disease transmission coefficients $\beta_h$ (vector-to-human; rate per mosquito per day), $\beta_V$ (human-to-vector; rate per human per day), $\Lambda_V$ (vector recruitment rate; vectors per day) and initial number of \textit{Exposed} humans ($E_H(0)$) by region} }
    \begin{tabular}{|l|c|c|c|c|} \hline
     \textbf{Region }  &  \textbf{$\beta_h$ estimate (S.D.)} & \textbf{$\beta_V$ estimate (S.D.)} & \textbf{$\Lambda_V$ estimate (S.D.)} & \textbf{$E_H(0)$ estimate (S.D.)}\\ \hline
     North    & 2.42e-12 (7.32e-14) & 7.99e-05 (2.68e-05) & 73463.80 (16974.80) & 66.29 (57.84)\\\hline
     Northeast &1.23e-12 (1.26e-14) & 2.84e-05 (2.78e-06)& 106358.00 (11505.10) & 3.10e-114 (2.03e-04)\\\hline
     Southeast&7.07e-10 (6.65e-11) & 1.60e-09 (1.80e-10)& 0.00 (27114.70) & 3980.72 (314.53)\\\hline
     South& 3.81e-13 (5.40e-15)& 1.80e-04 (5.31e-05)& 4.32e-17 (328.06) & 11.89 (9.94)\\\hline
     Central West&2.32e-09 (2.49e-09)& 1.10e-08 (1.39e-08)& 1272.11 (141029.00) & 2597.01 (121.31)\\\hline \hline
     Brazil & 2.48e-10 (4.68e-14)& 1.16e-09 (6.97e-11)& 0.00 (31594.70) & 6790.05 (557.26)\\\hline
    \end{tabular}
     \begin{flushleft}
    %  Contact authors for detailed reports
     \end{flushleft}
    
    \label{tab:zika_param_est}
    \end{adjustwidth}
\end{table}

% \begin{table}[]
%     \centering
%     \caption{\textbf{Best fit parameter estimates of vector recruitment rate $\Lambda_V$ and initial number of \textit{Exposed} humans $E_H(0)$}}
%     \begin{tabular}{|l|c|c|}\hline
%         \textbf{Region} & \textbf{$\Lambda_V$ estimate (S.D.)} & \textbf{$E_H(0)$ estimate (S.D.)} \\\hline
%          North & 73463.80 (16974.80) & 66.29 (57.84)\\\hline
%          Northeast & 106358.00 (11505.10) & 3.10e-114 (2.03e-04)\\\hline
%          Southeast & 0.00 (27114.70) & 3980.72 (314.53)\\\hline
%          South & 4.32e-17 (328.06) & 11.89 (9.94)\\\hline
%          Central West & 1272.11 (141029.00) & 2597.01 (121.31)\\\hline \hline
%          Brazil & 0.00 (31594.70) & 6790.05 (557.26)\\\hline
         
%     \end{tabular}
    
%     \label{tab:param-est-2}
% \end{table}

\begin{table}
    \centering
    \caption{\textbf{Population and Population Densities of Brazil by region}}
    \begin{tabular}{|l|p{1in}|p{1.4in}|p{1.38in}|} \hline
     \textbf{Region }  &  \textbf{Population (in millions)} & \textbf{Population Density (people/km$^2$)} & \textbf{Best-fit Vector Density at equilibrium (mosquitoes/km$^2$)}\\ \hline
     North    & 17.7 &4.6&47.69\\\hline
     Northeast &56.9&30.55&363.51\\\hline
     Southeast&15.6&77.96&930.67\\\hline
     South&86.3&43.46&509.34\\\hline
     Central West&29.4&7.2&96.76\\\hline \hline
     Brazil &205.26&24.1&240.92\\\hline
    \end{tabular}
    
    \label{tab:zika_pop}
\end{table}

\begin{figure}
     \centering
     \begin{subfigure}[b]{0.3\textwidth}
         \centering
         \includegraphics[width=\textwidth]{Zika_PE_figs/N.png}
         \caption{North}
         \label{fig:north_PE}
     \end{subfigure}
     \hfill
     \begin{subfigure}[b]{0.3\textwidth}
         \centering
         \includegraphics[width=\textwidth]{Zika_PE_figs/NE.png}
         \caption{Northeast}
         \label{fig:NE_PE}
     \end{subfigure}
     \hfill
     \begin{subfigure}[b]{0.3\textwidth}
         \centering
         \includegraphics[width=\textwidth]{Zika_PE_figs/SE.png}
         \caption{Southeast}
         \label{fig:SE_PE}
     \end{subfigure}
      \hfill
     \begin{subfigure}[b]{0.3\textwidth}
         \centering
         \includegraphics[width=\textwidth]{Zika_PE_figs/S.png}
         \caption{South}
         \label{fig:south_PE}
     \end{subfigure}
      \hfill
     \begin{subfigure}[b]{0.3\textwidth}
         \centering
         \includegraphics[width=\textwidth]{Zika_PE_figs/CW.png}
         \caption{Central West}
         \label{fig:CW_PE}
     \end{subfigure}
      \hfill
     \begin{subfigure}[b]{0.3\textwidth}
         \centering
         \includegraphics[width=\textwidth]{Zika_PE_figs/B.png}
         \caption{Brazil}
         \label{fig:brazil_PE}
     \end{subfigure}
        \caption{Model fit to data with best-fit parameters from Table \ref{tab:zika_param_est}. Error stands for the weighted error (residuals).}
        \label{fig:PE_figs}
\end{figure}

% \begin{figure}
% \centering
% \caption{\textbf{PRCC for the estimates of transmission coefficients $\beta_h$ (vector-to-human) and $\beta_V$ (human-to-vector) }. A: Population. B:	Population density. C:	Mosquito density. D:	Outbreak size. E: GDP (USD billion).}
% \begin{subfigure}[b]{0.45\textwidth}
%          \centering
%          \includegraphics[width=\textwidth]{Zika_PE_figs/beta_h_PRCC.jpg}
%          \caption{$\beta_h$ }
%          \label{fig:betah_prcc}
%      \end{subfigure}
%       \hfill
%      \begin{subfigure}[b]{0.45\textwidth}
%          \centering
%          \includegraphics[width=\textwidth]{Zika_PE_figs/beta_V_PRCC.jpg}
%          \caption{$\beta_V$}
%          \label{fig:betav_prcc}
%      \end{subfigure}
        
%         \label{fig:PRCC}
% \end{figure}

\section*{Discussion}
\paragraph{Limitations.} We assume that each region is isolated from the others during the calibration of the model for each region (that is, no movement of people or mosquitoes between the regions). No vertical transmission (since we simulate for only a year). No temperature dependence of mosquito-related processes. No vertical transmission.
\paragraph{Future work.} 

\section*{Conclusion}

\section*{Supporting information}

\paragraph*{S1 Fig.}
\label{S1_Fig}
{\bf Bold the title sentence.} Add descriptive text after the title of the item (optional).
\paragraph*{S1 File.}
\label{S1_File}
{\bf Lorem ipsum.}  Maecenas convallis mauris sit amet sem ultrices gravida. Etiam eget sapien nibh. Sed ac ipsum eget enim egestas ullamcorper nec euismod ligula. Curabitur fringilla pulvinar lectus consectetur pellentesque.

\paragraph*{S1 Video.}
\label{S1_Video}
{\bf Lorem ipsum.}  Maecenas convallis mauris sit amet sem ultrices gravida. Etiam eget sapien nibh. Sed ac ipsum eget enim egestas ullamcorper nec euismod ligula. Curabitur fringilla pulvinar lectus consectetur pellentesque.

\paragraph*{S1 Appendix.}
\label{S1_Appendix}
{\bf Lorem ipsum.} Maecenas convallis mauris sit amet sem ultrices gravida. Etiam eget sapien nibh. Sed ac ipsum eget enim egestas ullamcorper nec euismod ligula. Curabitur fringilla pulvinar lectus consectetur pellentesque.

\paragraph*{S1 Table.}
\label{S1_Table}
{\bf Lorem ipsum.} Maecenas convallis mauris sit amet sem ultrices gravida. Etiam eget sapien nibh. Sed ac ipsum eget enim egestas ullamcorper nec euismod ligula. Curabitur fringilla pulvinar lectus consectetur pellentesque.

\section*{Acknowledgments}

\nolinenumbers
\bibliographystyle{plos2015}
\bibliography{zika_ref}
\pagebreak
\appendix
\section{Data}
\begin{table}
    \centering
    \includegraphics[height = 0.85\textheight,  trim = 2cm 4cm 2cm 4cm]{Zika_PE_figs/Brazil_zika_by_region.pdf}
    \caption{Brazil data. Source: \url{http://tabnet.datasus.gov.br/cgi/tabcgi.exe?sinannet/cnv/zikabr.def} (Accessed: March 2021)}
    \label{tab:zika_data}
\end{table}

\textbf{Best fit total vector population at equilibrium:}

North: 1.75968e+08

Northeast: 5.6751e+08

Southeast: 8.63e+08

South: 2.94e+08

Central West: 1.55982e+08

Brazil: 2.0516e+09
\end{document}